\chapter{Method}
This chapter details the methods used in this project. First, the significance of the research is outlined against the context of the literature review. Next, the datasets used for training and validation of models are described and compared. Then, the ...

\section{Significance of this research}

\section{Data}
MovieLens, GoodReads, Netflix Prize, Jester

\section{Latent factor recommender models}
The main point of progression in within the field of collaborative filtering during the Netflix Prize was the shift from neighbourhood models to the use of latent factor models.

\subsection{Matrix factorisation}
Matrix factorisation provides a means for learning latent factor representations of user preferences. Furthermore, these models can include additional user and bias constant terms to accommodate any bias inherent in the feedback provided by users -- some users tend to rate items higher than others, and some items tend to be rated higher than others.

The use of these models helped to improve the state of the art in predictive ability in collaborative filtering problems; however, they include the assumption that ratings are produced as the linear combination of user and item latent factors. The advances in the field of deep learning over the last decade have shed light on the ability of neural networks to learn non-linear relationships in data sets...

\section{Embedding layers in neural networks}

\section{Model architecture}

\section{Evaluation and comparison of models}
