\chapter{Introduction}

\label{intro} % For referencing the chapter elsewhere, use \ref{intro} 

\section{Problem Description}
The purpose of a  recommender system is to help people make decisions in an area where they do not necessarily have a wealth of personal experience \parencite{rs_1.1_Resnick}. The output of a recommender system is usually in the form of a suggestion to the user, helping them to make decisions such as what items to buy, what music to listen to or what movie to rent \parencite{handbook_1.1_intro}.

With the internet boom of the 90s ... growth of e-commerce ... too many options started to overwhelm users. 

Already in 2007, the online video subscription service Netflix had collected nearly 2 billion ratings from more than 11.7 million subscribers on over 85 thousand titles in its first 10 years of operating \parencite{netflix_description}.

More choices means more feedback, we now have more recorded interactions of what people do -- and \textit{don't} -- like than ever before.

Much of these data are low-dimensional, typically containing only three attributes for each interaction.

Advances in deep learning, combined with the recent advent of GPU training, have unlocked the potential to retrieve cleaner signal from large data sets.

The Netflix Prize contest in 2008 paved the way for large improvements the state of the art for predictive ability in collaborative filtering models, but this was before the mainstream adoption of deep learning.

This project seeks to revisit the advances made by the contestants of the Netflix Prize and compare their results to those achievable with more modern deep learning approaches.

\section{Background to Research}


\section{Purpose of the Research}


\section{Layout of the Paper}
The literature review provides an overview of the relevant literature on recommender systems. The chapter covers the methodology surrounding the evaluation of recommender systems and outlines the two main types of models, namely content-based and collaborative filtering models. Particular attention is given to collaborative filtering models applied to explicit feedback data, leading to an in-depth discussion of the models that have been developed to perform this task on explicit feedback data.

Chapter 3 ...

Chapter 4 ...

Chapter 5 ...

Conclusions and recommendations for further research are outlined in Chapter 6


 